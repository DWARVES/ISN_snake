\documentclass{article}

\usepackage[utf8]{inputenc}
\usepackage[T1]{fontenc}
\usepackage[frenchb]{babel}
\usepackage[top=3cm, bottom=3cm, left=3cm, right=3cm]{geometry}
\usepackage{hyperref}
\usepackage{graphicx}
\usepackage{fancyhdr}
\usepackage{fancybox}

\hypersetup{colorlinks=true}

\title{Snakekans\\-- Projet ISN --}
\author{Luc Chabassier <luc.linux@mailoo.org> \and Pablo Donato <pablo.donato@mailoo.org>}

\begin{document}
\maketitle

\tableofcontents

\section{Introduction.}
Ce programme doit être créé dans le cadre d'un projet d'ISN en 2013. L'idée initiale est de faire un snake en réseau (deux joueurs).

\section{Fonctionnalités.}
Contrairement à l'idée initiale, le jeu est finalement capable de gérer jusqu'à quatre joueurs en local, en connectant des joysticks. L'idée du réseau est conservée : un des joueurs sera le serveurs, et donc effectuera la majorité des calculs, et des joueurs devront pouvoir se connecter à lui.

Le jeu s'organise sur une grille virtuelle composée de case de 20 pixels sur 20 pixels.

\subsection{Règles du jeu.}
Le programme est un snake classique, c'est à dire que le terrain ne contient que les serpents et les bonus. Quand un serpent sort d'un côté, il entre de l'autre. Les bonus apportent un certain nombre de points chacun. Le jeu se termine quand un des serpents meurt, c'est à dire qu'il se cogne contre un autre des serpents ou contre lui-même. Le gagnant est celui qui a le plus de points. Afin de dévaloriser la mort, chaque serpent a un malus en fonction du nombre de serpents restants quand il meurt.

S'il y a plusieurs joueurs, le jeu se termine quand n'y n'en reste plus qu'un vivant. S'il n'y a qu'un seul joueur, le jeu se termine à sa mort.

\subsection{Les bonus.}
Un bonus possède plusieurs attributs. Tout d'abord, le nombre de points qu'il rapporte et l'influence sur la taille : un bonus peut donc faire perdre des points et/ou raccourcir le serpent. Un bonus possède aussi une durée durant laquelle il reste affiché et une probabilité d'apparition. Enfin, chaque bonus possède une image propre.

Les bonus sont stocké dans un ficher config (clé=valeur) de la forme :
\begin{verbatim}
pts = 10
length = 3 # influence for the snake length
time = 10 # in seconds
picture = pictures/banana.png
fact = 20 # Higher is fact, higher is the apparition frequency
\end{verbatim}

Tous les fichiers des bonus sont dans un dossier dont le contenu est scanné au début du programme.

\subsection{Le serpent.}
Le serpent est stocké dans un structure. Il peut se déplacer dans quatre directions. Un serpent est affiché avec une succession d'images, chargée depuis un tileset. Il existe quatre image pour le serpent : la tête, un corp droit, un corp courbe et la queue. Chacune de ces images existe en deux exemplaire, qui servent à créer une animation lors du déplacement du serpent.

\subsection{Le terrain.}
Le terrain est constitué d'une image de fond, uniquement décorative. Le nombre de cases, ainsi que leur taille, est fixé dans le code. Chaque case non vide contient ou un bonus, ou une information de collision quand il y a le corp du serpent.

\section{Développement.}
\subsection{Bibliothèques utilisées.}
Ce jeu est codé en c++, et certaines bibliothèques SDL : tout d'abord la bibliothèque SDL de base, avec ses extensions SDL\_image, SDL\_ttf et SDL\_mixer. Certaines bibliothèque boost sont elles aussi utilisées : boost.filesystem et boost.regex.

Ce cahier des charges est rédigé en \LaTeX.

\subsection{Outils.}
L'outil utilisé pour écrire le code est Vim (éditeur et colorateur). Le code est versionné avec git (le code est hébergé sur github) et la chaine de compilation est générée par Cmake. Le jeu est dévellopé sous Linux mais devra être compatible avec d'autre plateforme comme Windows ou Mac.

\section{Licence.}
Le code est sous licence GNU GPLv3 (voir le fichier LICENCE fournit avec le code), au copyright de Luc Chabassier et Pablo Donato.

Les ressources graphiques (images, tileset) ont été crées par Luc Chabassier, et sont sous licence CC-BY-SA.

Les sons et musiques sont tirées du jeu SuperTux2, et sont sous licence CC-BY-SA.

\end{document}

